\documentclass[lettersize,journal]{IEEEtran}
\usepackage{amsmath,amsfonts}
\usepackage{algorithmic}
\usepackage{algorithm}
\usepackage{array}
\usepackage[caption=false,font=normalsize,labelfont=sf,textfont=sf]{subfig}
\usepackage{textcomp}
\usepackage{stfloats}
\usepackage{url}
\usepackage{verbatim}
\usepackage{graphicx}
\usepackage{cite}
\usepackage{xcolor}
\usepackage{parskip}
\hyphenation{op-tical net-works semi-conduc-tor IEEE-Xplore}
% updated with editorial comments 8/9/2021

\begin{document}

\title{An investigation into the effectiveness and impact of a Continuous Delivery pipeline upon university-level games development teams}

\author{Frost Donovan}

\maketitle

\begin{abstract}
    % What's the problem? What am I looking at? How does that help solve the problem? 
    % Opening, Challenge, Action, Resolution 

    Continuous Delivery (CD) is a technique designed to increase reliability and consistency with delivery of builds. This can then help to increase frequency of testing, the accountability of the development team, as well as the teams transpareny and \textit{"all being on the same page ness"}. This can increase team moral as well as stakeholder confidence, as both are able to regularly see the current state and rate of progress for the project. This encourages regular analysis of development pace and project scope, both internal and external. \\
    This stakeholder confidence and awareness of scope is a common problem within student development teams, so this investigation will answer to what extent the deployment of a CD pipeline will help to reduce these problems.
\end{abstract}

\section{Introduction}
    \subsection{What is Continuous Delivery?}
        CD is an expansion of Continuous Integration (CI), a pipeline for the continuous integration of code, being developed on separate branches, into the main branch. This code is merged into the main branch and then tested to ensure there are no merge conflicts or obvious errors thrown from the combined code. Assuming all of these tests pass, this merged code is pushed to the project repository. A key part of this pipeline is that the entire process is automated, ensuring minimal time is wasted waiting for tests to run or code to be uploaded. \cite{ContDelIntro,CICDCD} \\
        CD takes this one step further, in that after a project passes through the CI pipeline, the code is then compiled, packaged, and further tests that require the project to be compiled can be run. These tests should not be run before other tests that don't require compilation due to Fail Fast principles \cite{shore2004fail,bamboo}. Assuming there are no failures during compilation or testing, the built program can then be uploaded to somewhere where it can be easily accessed. Here, it is important to make a distinction between Continuous Delivery, where the build is available internally but not to users, and Continuous Deployment, where the build is pushed straight out to the current product users.
        core to agile methodology \cite{agilemanifesto}

    \subsection{Benefits \& Drawbacks of CD}
        The primary benefit of CD is reduced cycle time - a reduction in the time it takes for a change in the project to happen and then for the user to have that change applied to their version of the software. This principle is core to the agile methodology, being the very first principle in the agile manifesto \cite{agilemanifesto}. This faster cycle time means that feedback from active users can be obtained much quicker, both on the effectiveness of bug fixes and also on new features. Agile is designed to avoid the pitfalls of Waterfall \cite{royce1987managing}, one of which is the commitment of significant time and/ or resources into features that either are unattainable, or not actually wanted by the user. While other agile methods, such as Scrum or Extreme Programming \cite{cohen2004introduction,agilewithscrum}, can be an important part of the agile process, the effectiveness of a development team in delivering \textit{value} is always going to be dependant on the speed and reliability with which user feedback can be obtained. \\
        A benefit of this fast cycle time is not only are players able to see these changes faster, but stakeholders and publishers are able to see development progress, both regularly and on demand. This can be a significant step to building trust between a development studio and publisher, especially if the studio is new or doesn't have an existing relationship with the publisher \cite{gamedevhandbook}. 

        Part of the CD pipeline is testing, with a suite of unit tests being run on the code as part of the build process. As this build process is run consistently, rather than all at once leading up to a main release, this means bugs are found incrementally, stopping the accumulation of technical debt, reducing the cost to fix bugs, and reducing stress on programmers by preventing an overwhelming influx of bugs. While these unit tests will likely catch a lot of bugs, some bugs will only be caught during human playtesting. This decreased cycle time means that human playtesting can happen sooner \& more regularly, and fixes are delivered to testers \& players almost immediately, rather than having to wait for the next release window.
        
        Another strength of a Continuous Delivery pipeline is that it is fully automated. This allows less developer time to be spent setting up build or test environments and manually going through the build process, and more time on actually creating the product. This can be a \textit{significant} time save, with some large scale projects reportedly taking weeks to set up environments ready to produce a release build \cite{paddy, ContDelIntro}. This system also significantly reduces the chance that there are any errors caused by mistakes during the build process as this build-release pipeline will have had many iterations of the product pass through it, before a major release deadline. This increases the reliability and stability of new releases.
    
\section{Background \& Supporting Literature}
    With these benefits in mind it raises two questions; If a CD pipeline is this important and valuable, is it being taught to new developers in further education? If it is, is it actually as effective in practice with student teams as it is in theory?

    There is limited literature relating to Continuous Delivery being implemented in an academic context, and \textit{no} literature that I could find of this being implemented in a game development context. Even upon reviewing a literature review on rapid releases \cite{mantyla2015rapid} there were \textit{no} references to this within a games development context. This lack of literature provides a problem when attempting to find evidence on the performance of CD pipelines, however does highlight a \textit{need} for further literature and case studies on the subject. Given this lack, the papers being reviewed will be based around CD implementation in the academic setting of software development, a parallel field, but one which notably consists much more heavily of code, rather than the more even mix of skills and disciplines that is present within a game development context. As such, these papers all have a lack of analysis on how developers other than programmers respond to a CD pipeline, another case where there is a clear need for further study and literature.


    Has this been done before in academic setting\cite{CDCourse2014,CDMobileDev,IndustryAcademyDenmark}? 
    Links to other things - CI\cite{CICDCD}, Unit tests, regular product reviews, stakeholder (supervisor) confidence, git flow \cite{gitBranching}
    Best practices\cite{duvall2007continuous}? 

\section{Research Question}
    From the above sources, there is a clear need for research into the practical effects of a Continuous Delivery pipeline within game development and game development education. From this knowledge, I propose this initial investigation into the effects of a Continuous Delivery pipeline upon university-level games development teams, with a focus on the confidence of the team, as well as the confidence of the team's academic supervisor in their team. 
    This is purposefully broad, with the aim of encouraging and supporting further research into the topic. Given this, the following hypothesises will be investigated.

    \begin{itemize}
        \item The introduction of a CD pipeline will increase the confidence of the development team's supervisor.
        \item Access to a CD pipeline will increase the amount of playtesting a development team does
        \item Access to a CD pipeline will improve the team's scoping of the project
        \item A CD pipeline will increase the transparency of the project, giving developers a better understanding of the project's state and how their work contributes to that
    \end{itemize}

\section{Research Methodology}
    \subsection{Experimental Design}
        

    \subsection{Limitations}
        The primary limitation in this study is time. If this study were to be expanded, I would ideally like to follow team's from their first year all the way through to graduation, introducing half of the teams to the CD pipeline immediately and then tracking their results, feedback, and attitudes across all three years against the control groups. This could also be done across multiple cohorts, providing increased statistical significance and helping to account for random variation within team's cohesion and skill level. I could also foresee a technical and potentially ethical problem with this approach however.
        
        Technically, within Falmouth University Game's Academy, the department where these studies have taken place, teams are not the same each year, they are randomised. This would mean it would be impossible to track the same 'team', although potentially individuals could be tracked and seeing if they implemented CD pipelines of their own accord in future teams could also be a revealing statistic to look at, as it would likely give a measure of \textit{perceived} value, rather than actual value. \\
        The potential ethical problem could be if the initial results point towards a CD pipeline effecting a students grade, how is that dealt with. Does the experiment carry on as planned, potentially wilfully disadvantaging particular groups of students? While I am uncertain if this would be considered an ethical problem by an ethics board, it could likely provide a moral problem for those involved in running the study. A potential solution could be that those running the experiment would have to be kept in the dark about it's results while it was ongoing to prevent and bias or conflict of interest. While this would certainly require due consideration, it is not a concern for this initial study.

        Another limitation is that of resources available. Ideally, every team which a CD pipeline deployed would have support to help customise the pipeline, supporting the team in building custom processes and writing robust unit tests. This would likely give the most 'accurate' imitation of a CD pipeline under industry conditions, however we are unable to provide that level of support due to the research team being a singular individual whom has other commitments, and also lacks the experience to guide teams in writing unit tests specifically.
    
    \subsection{Sampling Plan}
        random sampling within Falmouth University Games Academy game's development teams. Aim is to sample as much of the population as possible, including team's across all three years. Mix of random and convenience sampling - 

        Sample size, sampling method
    
    \subsection{Data management plan}
        Data shall be collected using Microsoft Forms as it is GDPR compliant out the box. For any data that is then stored locally, it will be encrypted using GnuPG \cite{encryption} using a SHA256 cipher, with the unencrypted files being deleted using the 'shred' command on Linux.
        Data will be anonymised but matched after it has been collected, with participants names matching to unique ID numbers. The data will then be stored with this ID, rather than anything identifying like a name or email.
    
    \subsection{Data Analysis}
        T-test? Critical size for statistical significance?

    \subsection{Ethical Considerations}
        Due to the nature of this research, there are minimal ethical considerations that need to be taken into account. The participants will not be exposed to any potential risks outside of what they would experience under normal circumstances, although an effort has been made to keep the feedback form fairly condensed. This is to minimise any increased stress that the commitment of having to fill out the form may cause.
    
    \section{Appendix}
    Data analysis code, supporting screenshots, list of unit tests \& testing plan

% \section{Artifact}
%     \subsection{What will be made}
%         CD pipeline utilising Github Actions. Tool to set up secrets? Would be sick \url{https://docs.github.com/en/rest/reference/actions#secrets} \\
%         Continuous delivery or continuous deployment? Scope as deployment would need to include itch integration w/ butler, although it looks relatively simple to set up.
%         Autoupload to steam \& itch! \url{https://itch.io/docs/butler/}

%     \subsection{How will I ensure Quality}
%         Quality control. Roadmap? Unit Testing? Integration testing?

%     \subsection{How will I create it}

%     \subsection{Why will this answer the questions}

\bibliographystyle{ieeetr}
\bibliography{bibliography}

\end{document}